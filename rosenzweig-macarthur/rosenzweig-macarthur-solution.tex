% Meta
\documentclass[bulgarian, 12pt]{article}
\usepackage[
	a4paper, 
	includeheadfoot, 
	margin = 1.5 cm]
{geometry}

% Hyperlinks
\usepackage[
	unicode=true, 
	colorlinks=true, 
	linkcolor=black, 
	urlcolor=black]
{hyperref}

% Fonts
% UTF-8 if you are not using XeLaTeX
\usepackage[utf8]{inputenc}
\usepackage[T2A]{fontenc}
\usepackage[bulgarian]{babel}
\usepackage{bm}
% ISO-Math (only for XeLaTeX and LuaLaTex)
%\usepackage[math-style=ISO]{unicode-math}

% Citing
% IMPORTANT! USE 'babel=true' to be able to use csquotes with a multitude of languages
% By default you can use only ~10 languages
\usepackage[babel=true]{csquotes}

% Indent first line in paragraph
\usepackage{indentfirst}

% Place tags on the left
\usepackage[leqno]{amsmath}

% Better math
\usepackage{amssymb}
\usepackage{mathtools}
\usepackage{comment}
\usepackage{mathptmx}
\usepackage[makeroom]{cancel}

% Create math pictures
\usepackage{tikz}
\usepackage{enumitem}

% Better theorems
\usepackage{amsthm}

% Derivative notations
\usepackage{physics}
\usepackage{derivative}

%%%%%%%%%%%%%%%%%%%%%%%%%%%%%%%%%%%%%%%

% Lapacian delta
%\newcommand{\laplace}{\increment}
\newcommand{\laplace}{∆}
%\fontsize{16pt}{20pt}\selectfont
% Bolded cyrilic text
\renewcommand{\sfdefault}{cmss}
\renewcommand{\rmdefault}{cmr}
\renewcommand{\ttdefault}{cmt}

% Roman numerals for sections
\renewcommand{\thesection}{\Roman{section}} 
%\renewcommand{\thesubsection}{\thesection.\Roman{subsection}}
% Sectioning titles
\newtheorem{definition}{Дефиниция}[section]
\newtheorem{problem}{Задача}
\newtheorem{theorem}{Теорема}
\newtheorem*{theorem*}{Теорема}
\newtheorem{lemma}{Лема}
\newtheorem*{solution*}{Решение}

% Numbering of equations
\newcommand\numberthis{\addtocounter{equation}{1}\tag{\theequation}}

% Space between lines in array for fractions
\renewcommand{\arraystretch}{1.5}

\title{Решение на стационарна задача със самолетно крило}

\author{Калоян Стоилов}

\begin{document}

\maketitle

\begin{large}
\begin{equation}
\tag{D}
  \begin{cases}
    \pdv{N}{t} - \Delta N - N (1 - \frac{N}{0.4}) + \frac{0.5 N P}{0.5 + N} = 0, \quad \text{в $\Omega \times J
$} \\
    \pdv{P}{t} - \Delta P - \frac{0.5 N P}{0.5 + N} + 0.2 P = 0, \quad \text{в $\Omega \times J
$} \\
    \left(n \cdot \grad N \right) = \left(n \cdot \grad P \right)  = 0, \text{в $\partial \Omega \times J$} \\
    N(x,y,0) = 0.3 ,\quad P(x,y,0) = 0.5, \text{в $[0.8, 1]^2$} \\
    N(x,y,0) = P(x,y,0) = 0, \text{в $\partial \Omega$ при $t=0$}
  \end{cases}
\end{equation}

За да достигнем до вариационна формулировка, нека разгледаме за фиксиран момент $t \in J$ и умножим скаларно двете страни съответно по функции $v$ и $w$.
Ще приложим формулите за интегриране по части:
%\iint\limits_{\Omega} v \pdv{N}{t} \,\dd\Omega - \iint\limits_{\Omega} v  \Delta N \,\dd\Omega -\iint\limits_{\Omega} v N (1 - \frac{N}{0.4}) \,\dd\Omega + \iint\limits_{\Omega}  v \frac{0.5 N P}{0.5 + N} \,\dd\Omega = 0 \\

\begin{align*}
  &\iint\limits_{\Omega} v \left( \pdv{N}{t} - \Delta N - N (1 - \frac{N}{0.4}) + \frac{0.5 N P}{0.5 + N} \right) \,\dd\Omega = 0 \\
  &\iint\limits_{\Omega} v \pdv{N}{t} \,\dd\Omega = \dv{t} \iint\limits_{\Omega} v N \,\dd\Omega \\
  &\iint\limits_{\Omega} v (- \Delta N) \,\dd\Omega = \iint\limits_{\Omega} \grad v \cdot \grad N \,\dd\Omega - \int\limits_{\partial\Omega} v \cancelto{0}{\left(n \cdot \grad N \right)} \,\dd s \\
  &\iint\limits_{\Omega} w \left( \pdv{P}{t} - \Delta P - \frac{0.5 N P}{0.5 + N} + 0.2 P \right) \,\dd\Omega = 0 \\
  &\iint\limits_{\Omega} w \pdv{P}{t} \,\dd\Omega = \dv{t} \iint\limits_{\Omega} w P \,\dd\Omega \\
  &\iint\limits_{\Omega} w (- \Delta P) \,\dd\Omega = \iint\limits_{\Omega} \grad w \cdot \grad P \,\dd\Omega - \int\limits_{\partial\Omega} w \cancelto{0}{\left(n \cdot \grad P \right)} \,\dd s
\end{align*}
Имаме естествени гранични условия, откъдето няма условия над тестовите фунцкии: 
\[v, w \in H^1(\Omega)\]
Така достигнахме до вариационната задача:
\begin{align*}
\tag{V}
&\text{За всяко $t \in J$ търсим $N, P \in H^1(\Omega)$, такива че: \enspace} \\
&\forall v \in H^1(\Omega) \left(\dv{t} \iint\limits_{\Omega} v N \,\dd\Omega - \iint\limits_{\Omega} v N \,\dd\Omega + a(N, v) + b(N,v) + c(N,P,v) = 0 \right) \\
&\forall w \in H^1(\Omega) \left(\dv{t} \iint\limits_{\Omega} w P \,\dd\Omega + 0.2 \iint\limits_{\Omega} w P \,\dd\Omega + a(P, w) - c(N,P,w) = 0 \right) \\
&a(u, v) = \iint\limits_{\Omega} \grad v \cdot \grad u \,\dd\Omega ,\quad b(N, v) = \iint\limits_{\Omega} v \frac{N^2}{0.4} \,\dd\Omega ,\quad c(N, P, v) = \iint\limits_{\Omega} v \frac{0.5 N P}{0.5 + N} \,\dd\Omega
\end{align*}

Задачата на Риц-Гальоркин е: 
\begin{align*}
\tag{R-G}
&\text{За всяко $t \in J$ търсим $N_h, P_h \in H^1_h(\mathcal{K})$, такава че: \enspace} \\
&\forall v \in H^1_h(\mathcal{K} \left(\dv{t} \iint\limits_{\Omega} v N_h \,\dd\Omega - \iint\limits_{\Omega} v N_h \,\dd\Omega + a(N_h, v) + b(N_h,v) + c(N_h,P_h,v) = 0 \right) \\
&\forall w \in H^1_h(\mathcal{K}) \left(\dv{t} \iint\limits_{\Omega} w P_h \,\dd\Omega + 0.2 \iint\limits_{\Omega} w P_h \,\dd\Omega + a(P_h, w) - c(N_h,P_h,w) = 0 \right)
\end{align*}

Тук $\mathcal{K}$ е триангулацията. 
Както обикновено, нека базисните функции бележим $\varphi_i$. 
Тъй като тестовите функции са от едно пространство, то може да работим с един набор от базисни функции.
Така съответно и матрици на маса/коравина ще бъдат същите и може да се преизползват.

Нека с $\mathbf{q}, \mathbf{p}$ бележим съответно векторите от приближените стойности на $N$ и $P$ във възлите в момент $t$.
След стандартните разписвания имаме: 
\begin{align*}
  &\begin{cases}
	  M^0\dot{\mathbf{q}} + M^1\mathbf{q} + 2.5 B\mathbf{q} - = 0 \\
    M^0\dot{\mathbf{p}} + M^1\mathbf{p} + 0.2 M^0\mathbf{p} + = 0 \\
	\end{cases} \\
  &M^0=\begin{pmatrix}
	\iint\limits_{\Omega} \varphi_1 \varphi_1 \,\dd\Omega & \hdots & \iint\limits_{\Omega} \varphi_1 \varphi_n \,\dd\Omega \\
	\vdots & \ddots & \vdots \\
	\iint\limits_{\Omega} \varphi_n \varphi_1 \,\dd\Omega & \hdots & \iint\limits_{\Omega} \varphi_n \varphi_n \,\dd\Omega \\
	\end{pmatrix},\,
  M^1=\begin{pmatrix}
	\iint\limits_{\Omega} \grad \varphi_1 \cdot \grad \varphi_1 \,\dd\Omega & \hdots & \iint\limits_{\Omega} \grad \varphi_1 \cdot \grad \varphi_n \,\dd\Omega \\
	\vdots & \ddots & \vdots \\
	\iint\limits_{\Omega} \grad \varphi_n \cdot \grad \varphi_1 \,\dd\Omega & \hdots & \iint\limits_{\Omega} \grad \varphi_n \cdot \grad \varphi_n \,\dd\Omega \\
	\end{pmatrix}\\
  &B=\begin{pmatrix}
    \iint\limits_{\Omega} \varphi_1 \varphi_1^2 \,\dd\Omega & \hdots & \iint\limits_{\Omega} \varphi_1 \varphi_n^2 \,\dd\Omega \\
    \vdots & \ddots & \vdots \\
    \iint\limits_{\Omega} \varphi_n \varphi_1^2 \,\dd\Omega & \hdots & \iint\limits_{\Omega} \varphi_n \varphi_n^2 \,\dd\Omega \\
    \end{pmatrix}  
\end{align*}

$M^0, M^1$ могат да се сметнат като по лекции. За $B$ аналогично може да използваме локални матрици.
\end{large}
\end{document}