% Meta
\documentclass[bulgarian, 12pt]{article}
\usepackage[
	a4paper, 
	includeheadfoot, 
	margin = 1.5 cm]
{geometry}

% Hyperlinks
\usepackage[
	unicode=true, 
	colorlinks=true, 
	linkcolor=black, 
	urlcolor=black]
{hyperref}

% Fonts
\usepackage[T2A]{fontenc}
\usepackage[utf8]{inputenc}
\usepackage[bulgarian]{babel}
\usepackage{bm}
% ISO-Math (only for XeLaTeX and LuaLaTex)
%\usepackage[math-style=ISO]{unicode-math}

% Citing
% IMPORTANT! USE 'babel=true' to be able to use csquotes with a multitude of languages
% By default you can use only ~10 languages
\usepackage[babel=true]{csquotes}

% Indent first line in paragraph
\usepackage{indentfirst}

% Place tags on the left
\usepackage[leqno]{amsmath}

% Better math
\usepackage{amssymb}
\usepackage{mathtools}
\usepackage{comment}
\usepackage{mathptmx}
\usepackage[makeroom]{cancel}

% Create math pictures
\usepackage{tikz}
\usepackage{enumitem}

% Better theorems
\usepackage{amsthm}

% Derivative notations
\usepackage{physics}
\usepackage{derivative}

%%%%%%%%%%%%%%%%%%%%%%%%%%%%%%%%%%%%%%%

% Lapacian delta
%\newcommand{\laplace}{\increment}
\newcommand{\laplace}{∆}
%\fontsize{16pt}{20pt}\selectfont
% Bolded cyrilic text
\renewcommand{\sfdefault}{cmss}
\renewcommand{\rmdefault}{cmr}
\renewcommand{\ttdefault}{cmt}

% Roman numerals for sections
\renewcommand{\thesection}{\Roman{section}} 
%\renewcommand{\thesubsection}{\thesection.\Roman{subsection}}
% Sectioning titles
\newtheorem{definition}{Дефиниция}[section]
\newtheorem{problem}{Задача}
\newtheorem{theorem}{Теорема}
\newtheorem*{theorem*}{Теорема}
\newtheorem{lemma}{Лема}
\newtheorem*{solution*}{Решение}

% Numbering of equations
\newcommand\numberthis{\addtocounter{equation}{1}\tag{\theequation}}

% Space between lines in array for fractions
\renewcommand{\arraystretch}{1.5}

\title{Решение на стационарна задача със самолетно крило}

\author{Калоян Стоилов}

\begin{document}

\maketitle
\begin{large}
\begin{equation}
\tag{D}
    \begin{cases}
      - \Delta \Phi = 0, \quad \text{в $\Omega$} \\
      n \cdot \grad \Phi \mid_{\Gamma_{in}} = 1 \\
      u\mid_{\Gamma_{out}} = 0 \\
      n \cdot \grad \Phi \mid_{\Gamma_{else}} = 0 \\
    \end{cases}
\end{equation}

За да достигнем до вариационна формулировка, умножаваме скаларно двете страни по функция $v$ и прилагаме формула за интегриране по части:
\begin{align*}
-\iint\limits_{\Omega} v \Delta \Phi \,\dd\Omega &= 0 \\
-\iint\limits_{\Omega} v \left(\div \grad \Phi \right) \,\dd\Omega &= 0 \\
\iint\limits_{\Omega} \grad v \cdot \grad \Phi \,\dd\Omega\, - \int\limits_{\partial\Omega} \left(n \cdot \grad \Phi \right) v \,\dd s &= 0 \\
\iint\limits_{\Omega} \grad v \cdot \grad \Phi \,\dd\Omega\,  &= 
\int\limits_{\Gamma_{in}} \cancelto{1}{\left(n \cdot \grad \Phi \right)} v \,\dd s + \int\limits_{\Gamma_{out}} \left(n \cdot \grad \Phi \right) v \,\dd s + \int\limits_{\Gamma_{else}} \cancelto{0}{\left(n \cdot \grad \Phi \right)} v \,\dd s \\
\iint\limits_{\Omega} \grad v \cdot \grad \Phi \,\dd\Omega\,  &= 
\int\limits_{\Gamma_{in}} v \,\dd s + \int\limits_{\Gamma_{out}} \left(n \cdot \grad \Phi \right) \cancelto{0}{v} \,\dd s \\
\iint\limits_{\Omega} \grad v \cdot \grad \Phi \,\dd\Omega\,  &= 
\int\limits_{\Gamma_{in}} v \,\dd s \\
\end{align*}
За да подсигурим последното съкращаване, ще е необходимо: 
\[v \in V = \{v \in H^1(\Omega) \enspace\vert\enspace v\mid_{\Gamma_{out}}=0\}\]
Така достигнахме до вариационната задача:
\begin{align*}
\tag{V}
&\text{Търсим $\Phi \in V$, такава че: \enspace}
\forall v \in V \left(a(\Phi, v)=F(v)\right) \\
&a(\Phi, v) = \iint\limits_{\Omega} \grad v \cdot \grad \Phi \,\dd\Omega, \quad
F(v) = \int\limits_{\Gamma_{in}} v \,\dd s
\end{align*}

Билинейната форма за задачата ни е скаларно произведение и съответно може да приложим цялата теория. Задачата на Риц-Гальоркин е: 
\begin{align*}
\tag{R-G}
&\text{Търсим $\Phi_h \in V_h(\mathcal{K})$, такава че: \enspace}
\forall v \in V_h(\mathcal{K}) \left(a(\Phi_h, v)=F(v)\right)
\end{align*}

След стандартните разписвания имаме: 
\begin{align*}
	&M^1\mathbf{q}=\mathbf{b} \\
	&M^1=\begin{pmatrix}
	\iint\limits_{\Omega} \grad \varphi_1 \cdot \grad \varphi_1 \,\dd\Omega & \hdots & \iint\limits_{\Omega} \grad \varphi_1 \cdot \grad \varphi_n \,\dd\Omega \\
	\vdots & \ddots & \vdots \\
	\iint\limits_{\Omega} \grad \varphi_n \cdot \grad \varphi_1 \,\dd\Omega & \hdots & \iint\limits_{\Omega} \grad \varphi_n \cdot \grad \varphi_n \,\dd\Omega \\
	\end{pmatrix}, \quad
    \mathbf{b}=
    \begin{pmatrix}
      \int\limits_{\Gamma_{in}} \varphi_1 \,\dd s \\
      \hdots \\
      \int\limits_{\Gamma_{in}} \varphi_n \,\dd s \\
    \end{pmatrix}
\end{align*}
$M^1$ е сметната като по лекции. В кода се възползваме от факта, че при линейна интерполация градиента на фунцкиите на формата $\grad \boldsymbol{\Psi}$ е константен. Тогава директно може да се пресметне интеграла за локалните матрици на коравина, който е лицето на стандартния триъгълник. За $\mathbf{b}$ правим следното наблюдение - ако $i$ и $j$ са съседни възли по $\Gamma_{out}$, то отсечката $l$ от $\mathbf{r}_i=(x_i,y_i)^T$ до $\mathbf{r}_j=(x_j,y_j)^T$ е в $supp \,\varphi_i$ и $supp \,\varphi_j$. Нека разгледаме какво правим за $\varphi_i$:
\begin{align*}
&[0,1] \rightarrow (x,y)^T, \quad (x,y)^T \in l \\
&x(t) = t x_i + (1-t) x_j, \quad y(t) = t y_i + (1-t) y_j \\
&\dv{x}{t}\left(t\right) = x_i - x_j, \quad \dv{y}{t} \left(t\right) = y_i - y_j \\
&\int\limits_{l} \varphi_i \,\dd s = \int\limits_{0}^1 (1-t) \sqrt{\left(\dv{x}{t}\right)^2+\left(\dv{y}{t}\right)^2} \,\dd t = \norm{\mathbf{r}_i-\mathbf{r}_j}\int\limits_{0}^1 (1-t) \,\dd t \\ 
&= \norm{\mathbf{r}_i-\mathbf{r}_j}(1-\frac{1}{2}) =\frac{\norm{\mathbf{r}_i-\mathbf{r}_j}}{2} \\
\end{align*}
За получаването на $\int\limits_{\Gamma_{in}} \varphi_i \,\dd s$ остава само да съберем тези изрази за всички гранични отсечки, на които възела $i$ е край. Тъй като решението $\Phi$ е потенциал, то съответсващият му поток би бил $\pm \grad \Phi$ (в зависимост от кои източници следваме). Тук $-\grad \Phi$ изглежда повече приляга на задачата.

\end{large}
\end{document}
