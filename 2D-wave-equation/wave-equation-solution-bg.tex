% Meta
\documentclass[12pt]{article}
\usepackage[
	a4paper, 
	includeheadfoot, 
	margin = 1.5cm]
{geometry}

% Hyperlinks
\usepackage[
	unicode=true, 
	colorlinks=true, 
	linkcolor=black, 
	urlcolor=black]
{hyperref}

% Fonts
\usepackage[T2A]{fontenc}
\usepackage[utf8]{inputenc}
\usepackage[bulgarian]{babel}
\usepackage{csquotes}
\usepackage{bm}
% ISO-Math (only for XeLaTeX and LuaLaTex)
%\usepackage[math-style=ISO]{unicode-math}

% Indent first line in paragraph
\usepackage{indentfirst}

% Place tags on the left
\usepackage[leqno]{amsmath}

% Better math
\usepackage{amssymb}
\usepackage{mathtools}
\usepackage{comment}
\usepackage{mathptmx}
\usepackage[makeroom]{cancel}

% Create math pictures
\usepackage{tikz}
\usepackage{enumitem}

% Better theorems
\usepackage{amsthm}

% Derivative notations
\usepackage{physics}
\usepackage{derivative}

%%%%%%%%%%%%%%%%%%%%%%%%%%%%%%%%%%%%%%%

% Lapacian delta
%\newcommand{\laplace}{\increment}
\newcommand{\laplace}{∆}
%\fontsize{16pt}{20pt}\selectfont
% Bolded cyrilic text
\renewcommand{\sfdefault}{cmss}
\renewcommand{\rmdefault}{cmr}
\renewcommand{\ttdefault}{cmt}

% Roman numerals for sections
\renewcommand{\thesection}{\Roman{section}} 
%\renewcommand{\thesubsection}{\thesection.\Roman{subsection}}
% Sectioning titles
\newtheorem{definition}{Дефиниция}[section]
\newtheorem{problem}{Задача}
\newtheorem{theorem}{Теорема}
\newtheorem*{theorem*}{Теорема}
\newtheorem{lemma}{Лема}
\newtheorem*{solution*}{Решение}

% Numbering of equations
\newcommand\numberthis{\addtocounter{equation}{1}\tag{\theequation}}

% Space between lines in array for fractions
\renewcommand{\arraystretch}{1.5}

\title{Решение на стационарна задача със самолетно крило}

\author{Калоян Стоилов}

\begin{document}

\maketitle
\begin{large}
\begin{equation}
\tag{D}
    \begin{cases}
      \ddot u - \Delta u = 0, \quad \text{в $\Omega \times J
$} \\
      n \cdot \grad u \mid_{\Gamma_{N} \times J} = 0 \\
      u\mid_{\Gamma_{D} \times J} = 0.1 \sin (8\pi t) \\
      u = \dot u = 0, \text{в $\Omega$ при $t=0$} 
    \end{cases}
\end{equation}

За да достигнем до вариационна формулировка, нека разгледаме за фиксиран момент $t \in J$ и умножим скаларно двете страни по функция $v$ и приложим формула за интегриране по части:
\begin{align*}
\iint\limits_{\Omega} v \ddot u \,\dd\Omega -\iint\limits_{\Omega} v \Delta u \,\dd\Omega &= 0 \\
\iint\limits_{\Omega} v \ddot u \,\dd\Omega -\iint\limits_{\Omega} v \left(\div \grad u \right) \,\dd\Omega &= 0 \\
\iint\limits_{\Omega} v \ddot u \,\dd\Omega + \iint\limits_{\Omega} \grad v \cdot \grad u \,\dd\Omega\, - \int\limits_{\partial\Omega} \left(n \cdot \grad u \right) v \,\dd s &= 0 \\
\iint\limits_{\Omega} v \ddot u \,\dd\Omega + \iint\limits_{\Omega} \grad v \cdot \grad u \,\dd\Omega\,  &= 
\int\limits_{\Gamma_{N}} \cancelto{0}{\left(n \cdot \grad u \right)} v \,\dd s + \int\limits_{\Gamma_{D}} \left(n \cdot \grad u \right) v \,\dd s \\
\iint\limits_{\Omega} v \ddot u \,\dd\Omega + \iint\limits_{\Omega} \grad v \cdot \grad u \,\dd\Omega\,  &= 
\int\limits_{\Gamma_{D}} \left(n \cdot \grad u \right) \cancelto{0}{v} \,\dd s \\
\iint\limits_{\Omega} v \ddot u \,\dd\Omega + \iint\limits_{\Omega} \grad v \cdot \grad u \,\dd\Omega\,  &= 0 \\
\end{align*}
За да подсигурим последното съкращаване, ще е необходимо: 
\[v \in V = \{v \in H^1(\Omega) \enspace\vert\enspace v\mid_{\Gamma_{D}}=0\}\]
Така достигнахме до вариационната задача:
\begin{align*}
\tag{V}
&\text{За всяко $t \in J$ търсим $u \in V$, такава че: \enspace} \\
&\forall v \in V \left(\dv[2]{t} \iint\limits_{\Omega} v u \,\dd\Omega + a(u, v)=0\right) \\
&a(u, v) = \iint\limits_{\Omega} \grad v \cdot \grad u \,\dd\Omega
\end{align*}

Билинейната форма за задачата ни е скаларно произведение и съответно може да приложим цялата теория. Задачата на Риц-Гальоркин е: 
\begin{align*}
\tag{R-G}
&\text{За всяко $t \in J$ търсим $u_h \in V_h(\mathcal{K})$, такава че: \enspace} \\
&\forall v \in V_h(\mathcal{K}) \left(\dv[2]{t} \iint\limits_{\Omega} v u_h \,\dd\Omega + a(u_h, v) = 0\right)
\end{align*}

След стандартните разписвания имаме: 
\begin{align*}
	&\dv[2]{t} M^0\mathbf{q} + M^1\mathbf{q} = 0 \\
	&M^0\dv[2]{\mathbf{q}}{t} + M^1\mathbf{q} = 0 \\
	&M^0=\begin{pmatrix}
	\iint\limits_{\Omega} \varphi_1 \varphi_1 \,\dd\Omega & \hdots & \iint\limits_{\Omega} \varphi_1 \varphi_n \,\dd\Omega \\
	\vdots & \ddots & \vdots \\
	\iint\limits_{\Omega} \varphi_n \varphi_1 \,\dd\Omega & \hdots & \iint\limits_{\Omega} \varphi_n \varphi_n \\
	\end{pmatrix}, \quad
	M^1=\begin{pmatrix}
	\iint\limits_{\Omega} \grad \varphi_1 \cdot \grad \varphi_1 \,\dd\Omega & \hdots & \iint\limits_{\Omega} \grad \varphi_1 \cdot \grad \varphi_n \,\dd\Omega \\
	\vdots & \ddots & \vdots \\
	\iint\limits_{\Omega} \grad \varphi_n \cdot \grad \varphi_1 \,\dd\Omega & \hdots & \iint\limits_{\Omega} \grad \varphi_n \cdot \grad \varphi_n \\
	\end{pmatrix}
\end{align*}
$M^0, M^1$ са сметнати като по лекции. В кода се възползваме от факта, че при линейна интерполация градиента на фунцкиите на формата $\grad \boldsymbol{\Psi}$ е константен. Тогава директно може да се пресметне интеграла за локалните матрици на коравина, който е лицето на стандартния триъгълник. Нормализираната система изглежда така:
\begin{equation}
    \begin{cases}
      \dot {\mathbf{q}} = {\mathbf{p}} \\
      \dot {\mathbf{p}} = -(M^0)^{-1} M^1 {\mathbf{q}} \\
      \mathbf{q}(0) = \mathbf{p}(0) = 0 \\
    \end{cases}
\end{equation}
$M^0$ е силно разредена и ще приближим $(M^0)^{-1}$ с $(\bar{M}^0)^{-1}$, където с $\bar{M}^0$ сме означили концентрираната матрица, съотвестваща на $M^0$. За решаване с подобреният метод на Ойлер, то:
\begin{equation}
    \begin{cases}
      \mathbf{Q}_{j + 1} - \mathbf{Q}_j = \frac{\tau}{2} \left( \mathbf{P}_{j + 1} + \mathbf{P}_j \right) \\
      \mathbf{P}_{j + 1} - \mathbf{P}_j = -\frac{\tau}{2} (M^0)^{-1} M^1 \left( \mathbf{Q}_{j + 1} + \mathbf{Q}_j \right) \\
      \mathbf{Q}_0 = \mathbf{P}_0 = 0 \\
    \end{cases}
\end{equation}
За да се освободим от неявните пресмятания, може да направим следното:
\begin{align*}
&\mathbf{Q}_{j + 1} = \mathbf{Q}_j + \frac{\tau}{2} \left( 2\mathbf{P}_j -\frac{\tau}{2} (M^0)^{-1} M^1 \left( \mathbf{Q}_{j + 1} + \mathbf{Q}_j \right) \right) \\
&\mathbf{Q}_{j + 1} + \frac{\tau^2}{4} (M^0)^{-1} M^1 \mathbf{Q}_{j+1} = \mathbf{Q}_j - \frac{\tau^2}{4} (M^0)^{-1} M^1 \mathbf{Q}_j + \frac{\tau}{2} 2\mathbf{P}_j \\
&\left(I + \frac{\tau^2}{4} (M^0)^{-1} M^1 \right)\mathbf{Q}_{j + 1} = \left(I - \frac{\tau^2}{4} (M^0)^{-1} M^1 \right) \mathbf{Q}_j + \tau \mathbf{P}_j \\
&\mathbf{Q}_{j + 1} = \left(I + \frac{\tau^2}{4} (M^0)^{-1} M^1 \right)^{-1} \left(I - \frac{\tau^2}{4} (M^0)^{-1} M^1 \right) \mathbf{Q}_j + \tau \left(I + \frac{\tau^2}{4} (M^0)^{-1} M^1 \right)^{-1} \mathbf{P}_j \\
&\mathbf{Q}_{j + 1} = \left(I + A \right)^{-1} \left(I - A \right) \mathbf{Q}_j + \tau \left(I + A \right)^{-1} \mathbf{P}_j, \quad A = \frac{\tau^2}{4} (M^0)^{-1} M^1 \\
&\mathbf{P}_{j + 1} = \mathbf{P}_j -\frac{\tau}{2}  (M^0)^{-1} M^1 \left(\left(I + A \right)^{-1} \left(I - A \right) \mathbf{Q}_j + \tau \left(I + A \right)^{-1} \mathbf{P}_j + \mathbf{Q}_j \right) \\
&\mathbf{P}_{j + 1} = \mathbf{P}_j -\frac{2}{\tau} A \left(\left(I + A \right)^{-1} \left(I - A \right) \mathbf{Q}_j + \tau \left(I + A \right)^{-1} \mathbf{P}_j + \mathbf{Q}_j \right) \\
&\mathbf{P}_{j + 1} = \mathbf{P}_j -2 A \left(I + A \right)^{-1} \mathbf{P}_j - \frac{2}{\tau} A \left(\left(I + A \right)^{-1} \left(I - A \right) + I \right) \mathbf{Q}_j \\
&\mathbf{P}_{j + 1} = \left(I -2 A \left(I + A \right)^{-1}\right) \mathbf{P}_j - \frac{2}{\tau} A \left(\left(I + A \right)^{-1} \left(I - A \right) + I \right) \mathbf{Q}_j \\
\end{align*}
Ако въведем означанията:
\begin{equation}
B = I - A, \quad C = (I + A)^{-1}, \quad D = I - 2 A C, \quad E = C B, \quad F = -\frac{2}{\tau} A - \frac{2}{\tau} A E, \quad G = \tau C \\  
\end{equation}
Получаваме директната рекурентна зависимост:
\begin{equation}
	\begin{cases}
      \mathbf{Q}_{j + 1} = E\mathbf{Q}_j + G\mathbf{P}_j \\
      \mathbf{P}_{j + 1} = D\mathbf{P}_j + F\mathbf{Q}_j \\
      \mathbf{Q}_0 = \mathbf{P}_0 = 0 \\
    \end{cases}
\end{equation}
Но през цялото време си затворихме очите, че условието на Дирихле не е хомогенно. Затова тук след всяка стъпка пресмятания, ще задаваме компонентите на $\mathbf{Q}_{j + 1}$, съотвестващи на възлите по границата $\Gamma_{D}$ да са $\sin(8\pi t_{j+1})$
\end{large}
\end{document}
